\documentclass[11pt,a4paper]{article}
\usepackage[left=2cm,top=3cm,total={17cm,24cm}]{geometry}
\usepackage{times}
\usepackage[utf8]{inputenc}
\usepackage[IL2]{fontenc}
\usepackage[czech]{babel}
\usepackage{xurl}

\bibliographystyle{czplain}
\begin{document}
\begin{titlepage}
\begin{center}
    \Huge{\textsc{Vysoké učení technické v~Brně \\}}
    \huge{\textsc{Fakulta informačních technologií \\}}
    \vspace{\stretch{0.4}}
    \LARGE{Typografie a publikování\,--\,4. projekt\\}
    \Huge{Bibliografické citace}
    \vspace{\stretch{0.6}}
\end{center}
{\Large \today \hfill Hochman Martin}
\end{titlepage}

\section*{Úvod}
Typografie má začátky na přelomu 15. století, kdy vynálezce \textit{Johannes Gutengerg} uvedl do provozu první knihtisk, jež byl vyroben z~dřevěných matric. Od té doby je typografie běžnou součástí tvorby a zpracování jakýchkoliv textů. Velký posun v~této oblasti přišel s~příchodem osobních počítačů.\cite{Rybicka}

Užitečné informace nebo ukázku typografie z~přelomů 20. století můžete najít v~článcích \cite{Typografia-1} \cite{Typografia-2}.

\section*{Co je to \LaTeX}
\LaTeX je komplexní sada příkazů, která využívá sázecí program \TeX, což nám dovoluje připravovat veliké množství dokumentů, od různých článků, prezentací, až ke knihám. \LaTeX a \TeX jsou open source softwary, tudíž je můžeme využívat zdarma nebo tento software můžeme upravit a distribuovat pod jiným jménem. Podrobné rozdíly mezi systémy \LaTeX\ a \TeX\ jsou zmíněny v~\cite{Kopkac}.
Typografie by měla podporovat pointu textu a nikoliv ji přebíjet. \cite{Butterick}

\section*{Práce s~\LaTeX em}
\LaTeX se od většiny typografických nástrojů liší především v~tom, že jednotlivé úpravy textu sázíme pomocí příkazů. Což lehce může připomínat programování / kódování. Pro začátečníky je k~dispozici mnoho elektronických dokumentů \cite{Brabec} \cite{Overleaf}.

Další možností, kde najdeme hodně tipů a triků, je zakoupení časopisu \cite{Typo}.
\section*{\LaTeX\ na Vysoké škole}
Při psaní různých dokumentů do školy, či už technických zpráv nebo dokumentací k~projektům, většina studentů se znalostí \LaTeX u volí tuto možnost, jelikož nám to umožňuje napsat práci kvalitněji a rychleji.

V~závěru bakalářského studia je potřeba napsat bakalářskou práci právě ve zmiňovaném \LaTeX u, proto některé studenty právě téma \uv{typografie} nadchlo a vypracovali práci na toto téma, zde se můžeme podívat na některé z~nich \cite{Bakalarka} \cite{Diplomka}. 

\section*{Závěr}
Naučit se v~\LaTeX vyžaduje čas a chuť se učit stále novým věcem, protože typografie se každým dnem nadále vyvíjí a přizpůsobuje okolnímu světu. 

\newpage
\renewcommand{\refname}{Reference}
\bibliography{citace}
\end{document}
